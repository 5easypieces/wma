\documentclass{report}

\title{ Western Museums Association Program Committee: Session Proposals (Round 2)}
\date{ Last updated: August 13, 2020}
\author{Western Museums Association}

\begin{document}
  \pagenumbering{gobble}
  \maketitle
  \newpage
  \tableofcontents
  \newpage
  \pagenumbering{arabic}
  
    \newpage
    \chapter*{ Regular session/panel (roundtable, single speaker, etc.) }

      
        
          \newpage
          \section{ Preserving Grief: From Spontaneous Tributes to Permanent Memorials }
            \begin{description}
              \item [ID:]
              WMA2020\_SP026

              \item [Assigned to:]Angela Neller~
                \item [Track:]Collections~
              \end{description}

              This session will show how three institutions dealt with the challenges created by one event, the mass shooting in Las Vegas in October 2017, Panelists will discuss making the initial decision to collect items documenting the event, processing and exhibiting  difficult and emotional collections, and working with the community to create a permanent memorial. Panelists will also show how their responsibilities have changed with the changing needs of the community in the ensuing years.

              \subsection*{Session Information}
                \begin{description}
                  \item [Format:] Regular session/panel (roundtable, single speaker, etc.)
							    
							    \item [Uniqueness:]Having dealt with the increasingly relevant issue of documenting the aftermath of tragedy, now three years on, the panelists have unique perspectives to share.
							    \item [Objectives:]This session will show how staff at Las Vegas institutions made decisions to collect items documenting a local mass shooting in October, 2017, and how those decisions affected the way the tragedy is interpreted and remembered. It is aimed at collections management staff, curatorial staff, and management at museums and cultural institutions.  Attendees will be hear about many issues, both practical and philosophical, that can arise when collecting and interpreting current, tragic, events. These issues include: what should be collected from the actual event, from memorials, and from the community; when should this collecting take place, and when should it end; how should the items be used once they are part of an institution’s collection; what issues will institutions face in dealing with government officials, the media, and the community; what financial burdens will institutions face and how can additional funding be found; how will institutions be involved with permanent memorialization of an event. Attendees will also learn how institutions’ roles change, both in the immediate aftermath of a tragedy, and as time passes. Attendees will receive tips and strategies for how to deal with collecting after mass tragedies so that they can help their institutions prepare for a similar event. They will also learn about resources available from other institutions who have faced collecting after tragedies, particularly mass shootings. Presenters will provide written procedures created specifically for rapid response collecting as well as examples of press releases and other public outreach.
							    \item [Engagement:]Each panelist will present their experiences, and invite the audience to join in a discussion about what was presented.  If necessary, participation will be encouraged by asking audience members about their own experiences or their institutions’ plans to address issues covered in the presentations.  Since the panelists will have presentations, a projector and screen, or similar, will be needed.
							    \item [Relationship to Theme:]When collecting and interpreting tragic current events, institutions must learn to use their collections and expertise in new and unexpected ways. By preparing for changing opportunities and responsibilities, institutions can not only move themselves forward, but help the community move forward as well.
							    
                \end{description}
              \subsection*{Audience}
                \begin{description}
                  \item [Audiences:]Curators/Scientists/Historians~Registrars, Collections Managers~
                  \item[Professional Level:]Emerging Professional~Mid-Career~Senior Level~
                \item[Scalability:] Institutions of all sizes and types have been asked to collect and interpret spontaneous memorials. The information in this session will pertain to museums large and small, as well as libraries, archives, community centers, and local and regional government organizations.

							
              \end{description}
            \subsection*{Participants}
              \subsubsection*{ Cynthia Sanford }
              Submitter, Moderator, Presenter\newline
              Registrar\newline
              Clark County Museum, Henderson, Nevada
              \newline
              cynthia.sanford@clarkcountynv.gov\newline
              
              702-455-7955\newline

              In addition to presenting, I will introduce the other panelists and ask prepared questions to begin discussion and facilitate a question and answer period with questions from and to the audience.\newline


              

              
                \subsubsection*{ Caroline Kunioka }
                Presenter\newline
                Curator of History and Collections\newline
                Nevada State Museum Las Vegas, Las Vegas, Nevada
                \newline
                ckunioka@nevadaculture.org\newline
                
                702-822-8763\newline

                Caroline led her museum's response to the mass shooting in Las Vegas. She not only worked with other institutions in the area, but also contacted related corporations, organizations, and individuals in attempts to create a meaningful collection documenting the community's reaction to the event.
                \emph{ (confirmed) }
              

              
                \subsubsection*{ Mickey Sprott }
                Presenter\newline
                Public Art Supervisor\newline
                Clark County Parks and Recreation Department, Las Vegas, Nevada
                \newline
                Mickey1@clarkcountynv.gov\newline
                
                702-455-8685\newline

                Mickey worked with government officials, multiple museums and arts organizations, individual artists, and family members of victims in order to create an exhibit commemorating the one year anniversary of the mass shooting in Las Vegas. She is also working with a committee appointed to create a permanent memorial to the victims of the shooting.
                \emph{ (confirmed) }
              

              
                \subsubsection*{ Cynthia Sanford }
                Presenter\newline
                Registrar\newline
                Clark County Museum, Henderson, Nevada
                \newline
                cynthia.sanford@clarkcountynv.gov\newline
                
                702-455-7955\newline

                Cynthia organized and oversaw the intake, cataloging, and storage of thousands of items left at spontaneous memorials created after the mass shooting in Las Vegas. She also acted as the museum's liaison with those affected by the shooting, the media, and the general public.
                \emph{ (confirmed) }
              

              
        
          \newpage
          \section{ Annual Exhibit Critique: Oregon Historical Society’s Experience Oregon   }
            \begin{description}
              \item [ID:]
              WMA2020\_SP029

              \item [Assigned to:]Kathleen Daly~
                \item [Track:]
              \end{description}

              Join colleagues exploring current best practices in creating exhibitions—this year, with the Oregon Historical Society museum’s new permanent exhibit, Experience Oregon. This overarching history exhibit, incorporates opportunities to share ideas/opinions on relevant themes, interactive places throughout, and “Across Time” stations that use broad themes to draw connections between yesterday and today; emphasizing why learning about history matters. The museum is an evening event location, and is free to visit during the conference; please try to see the exhibit prior to the session.

              \subsection*{Session Information}
                \begin{description}
                  \item [Format:] Regular session/panel (roundtable, single speaker, etc.)
							    
							    \item [Uniqueness:]This session provides a forum for multiple perspectives from a variety of museum professionals to be expressed, and facilitates increasing our field’s creative tools.
							    \item [Objectives:]The session aims, through constructive critique and dialogue, to inform audience  members about ways of meeting challenges encountered while creating, installing, and presenting exhibits -- issues for exhibit practice that the exhibit raises. Attendees at this session will engage with curatorial practice and visitor engagement through discussions revolving around best practices in exhibit work.   In addition to critiquing the single exhibit, the staff at OHM is also interested in hearing impressions from the critique panel and the audience about the new exhibit and experiences at OHM, to incorporate as they design their next experiences. Takeaways will be the notes (mental or written) that audience members make as they listen and participate.  Spontaneity is a hallmark of the session, so the specific takeaways cannot be predicted in advance. Past sessions have generated new ideas and approaches especially for operationally bringing exhibits from idea to reality.   Tangible takeaways include 1) handouts outlining OHM’s exhibit development process, 2) handouts with information and links to current resources on exhibit design, and 3) information on best practices in community engagement through exhibit design.  Finally, as the session organizer, two of my goals in taking over the session planning in 2018 has been to expand the pool of Critiquers, and to bring in new voices to the panel.
							    \item [Engagement:]Short intro to the panelists/ session;  Short visual introduction to the exhibit (so anyone one who didn’t see the exhibit can get an idea of the layout);  A critiquer presents overall thoughts, impressions, critique, and questions;  Two critiquers pair up - one reviewer/one exhibit team member – for conversations/fire side chats, calling out interesting and/or challenging aspects of the exhibit from different professional lenses and the experience going through the exhibit;  Leaves about 30 minutes for audience questions/ audience-driven conversation.
							    \item [Relationship to Theme:]Few things in museums move us FORWARD more than professionals in the field sharing ideas, in this case about exhibits. The ability for audience members to hone their exhibit critiquing skills, while learning from experiences of other professionals in the field, as we examine current exhibition practice, should leave all attending thinking about how they might apply what they learn to move their exhibits and institutions FORWARD.
							    
                    \item [Additional Comments: ]Session Abstract: Creating an exhibit, in any discipline, is not a task for the faint of heart. It takes considerable vision, collaboration, planning, and effort to craft an exhibit with a clear purpose, inclusive narrative, and broad engagement. Each year, the WMA annual exhibit critique session shares out observations and assessments of a current exhibit. These have come from a wide variety of museum types, and have covered a range of topics, with the overarching goals being to help the field recognize excellence, create critical dialogue, and to engage museum professionals at WMA in a reflexive process of analysis of exhibit practice in the 21st century. The session focuses on critique that is analytical rather than descriptive. We review large exhibitions seen by thousands of people, small innovative exhibits, and exhibits focused on underrepresented topics. The “critiquers” feature regional museum professionals, usually curators, exhibit developers/designers, educators, engagement, and/or collections folks, who each critique the exhibit from their own professional perspective - what works well, areas they see that didn't work as well or that confused them, innovative ideas the exhibit highlights (could be tech, content, presentation, design, programming, etc.), and questions they have for the exhibit creation team. Select members of site museum’s exhibit creation team has opportunities to share their exhibit, the ideas/intentions/goals behind it, what they like best about it, what they would do differently if they could, next steps if any, and answer the reviewers questions in dialogue with them and the audience.   

                \end{description}
              \subsection*{Audience}
                \begin{description}
                  \item [Audiences:]Curators/Scientists/Historians~
                  \item[Professional Level:]Emerging Professional~General Audience~Mid-Career~Student~
                \item[Scalability:] Audience members should be able to apply lessons learned and shared to any size museum exhibit. The handouts on handouts outlining OHM’s exhibit development process, and those handouts with information and links to current resources on exhibit design will be useful to all.

							
              \end{description}
            \subsection*{Participants}
              \subsubsection*{ Keni Sturgeon }
              Submitter, Moderator\newline
              Executive Director\newline
              Wenatchee Valley Museum \& Cultural Center, Wenatchee
              \newline
              ksturgeon@wvmcc.org\newline
              ksturgeon@wvmcc.org\newline
              5098886242\newline

              Moderator/Organizer. I'll also ensure balance of time and questions from the audience, so one individual cannot dominate the Q\&A/Discussion. I took over organizing this annual session in 2018.  Previously I served as an Exhibit Critiquer in this annual session for 3 years. I have more than 12 years of exhibit design, installation and curation experience, and have worked in visitor engagement in zoos, aquaria, history and cultural museums, and science centers for more than 20 years.\newline


              

              
                \subsubsection*{ Seth  Margolis }
                Presenter\newline
                Director of Education Programs\newline
                Musuem of Flight, Seattle, WA
                \newline
                SMargolis@museumofflight.org\newline
                
                (206) 768-7116\newline

                Seth is the Director of the Education Department at The Museum of Flight in Seattle. He studied history at the University of Alberta, received his MA in Museology at the University of Washington (UW). Seth has worked in museums in the U.S. and Canada. He also museum education for the UW’s Graduate Program in Museology and serves on the advisory board for the Museum Studies Certificate Program. Seth has developed exhibit content, programming, and hands-on engagement activities for numerous history-based museums.
                \emph{ (confirmed) }
              

              
                \subsubsection*{ Kate  Fernandez }
                Presenter\newline
                Director of Interpretation \& Visitor Experience\newline
                Burke Museum, Seattle, WA
                \newline
                kfern@uw.edu\newline
                
                206.685.1731   \newline

                Kate is the director of interpretation \& visitor experience at the Burke Museum, overseeing the strategy and implementation of exhibits, programs and the visitor experience team. Previously, she developed award-winning exhibits about local communities for MOHAI in Seattle’s Museum of History \& Industry. Kate leads her exhibit team with a strong design sense and she brings an understanding of audience to all of her work. She holds a degree in Comparative History of Ideas and a minor in American Indian Studies from the University of Washington and a Certificate in Museum Studies from UW Continuum College.
                \emph{ (confirmed) }
              

              
                \subsubsection*{ Lorie Millward }
                Presenter\newline
                VP of Possibilities\newline
                Thanksgiving Point Institute, Lehi, UT 
                \newline
                lmillward@thanksgivingpoint.org\newline
                
                (801) 768-2300 \newline

                As the VP of Possibilities, Lorie is responsible for the entire spectrum of experience programming, including: exhibition, audience research, design, education, grant management, venue content specialists, and volunteers for Thanksgiving Point. Cultural institutions, as places where people visit because they choose to learn, can and should be places where all feel welcomed, valued, and challenged to be better humans, which is why Lorie has given 30 years of her life to cultural institutions as a humble learner, leader, and advocate striving to spread the gospel of free-choice learning, self-discovery, and personal empowerment.
                \emph{ (confirmed) }
              

              
                \subsubsection*{ Tara Cole }
                Presenter\newline
                Museum Services Coordinator\newline
                 Oregon Historical Society, Portland, OR
                \newline
                Tara.Cole@OHS.org\newline
                
                (503) 222-1741\newline

                Tara was part of the exhibit team that created and installed Experience Oregon.
                \emph{ (confirmed) }
              
        
          \newpage
          \section{ Working Together: An Inclusive Approach to Exhibit Planning and Design }
            \begin{description}
              \item [ID:]
              WMA2020\_SP043

              \item [Assigned to:]Cory Gooch~
                \item [Track:]
              \end{description}

              How can museums target, work with, and attract a more diverse audience? This session explores how the Oregon Historical Society created their core exhibit Experience Oregon using participatory models (co-creation, collaboration) for exhibit development. This session is an open forum – a back-and-forth dialog between audience and panelists. Attendees will leave with steps and strategies on how to tell a more complete story, work with multiple groups, address conflicting perspectives, navigating roadblocks, and lesson learned.

              \subsection*{Session Information}
                \begin{description}
                  \item [Format:] Regular session/panel (roundtable, single speaker, etc.)
							    
							    \item [Uniqueness:]This session delves into the strengths and challenges of using participatory models – co-creation and collaboration – to develop an exhibit grounded in multiple perspectives.
							    \item [Objectives:]This session is appropriate for a general audience at all professional levels. The session is an open forum – a back-and-forth dialog between audience and panelists on the exhibit-design process.  Panelists will speak briefly to introduce their role in the project before opening the floor for questions.  The objectives and learning outcomes address roadblocks (including internal stakeholders), unexpected finds, and lessons learned (including negative feedback) while developing Experience Oregon.  1.) Community engagement and management of stakeholder group.  The session will discuss the importance of community engagement and working with diverse stakeholders.  Corralling the various groups requires tenacity, patience, strategic planning, and constant follow through. Discussions will explore what worked, and what didn’t work. 2.) How to incorporate multiple perspectives and differing viewpoints.  Embracing multiple perspectives in a project can, and will, lead to conflicting viewpoints on interpretation. The audience should come away with insights on how to navigate differing viewpoints while including multiple perspectives.  3.) How to respond to, or not respond to, negative feedback. Presenting a fuller, more complete narrative of history (read – the good, the bad, and the ugly) may be challenging for visitors, stakeholders, or staff. One objective of this session is to talk about negative feedback, and some strategies to manage it. This session will provide an open, safe place to discuss the challenges of a participatory exhibit design process. Our goal is to provide concrete tools to apply this model within their institutions.
							    \item [Engagement:]This session will be fueled by the audience’s engagement in the discussions. There will be a brief presentation by the panelists before opening the floor for questions on the content development process, challenges faced, and lessons learned. Tangible takeaways include: a scalable, collaborative model for other institutions to follow, professional development, and hard earned approaches to consider prior to beginning a similar project.   Resources needed: Computer for powerpoint presentation, projector, and screen
							    \item [Relationship to Theme:]For museums and other historic institutions to be forward-thinking and adaptable, recognizing the need to include previously unheard voices in their exhibitions is paramount. Developing Experience Oregon was a co-creative, collaborative undertaking with participants from OHS staff, Oregon Tribes; educators; content specialists; students; historians; community members; and multiple firms from across the country. The exhibition embraces diversity, equity, inclusivity and accessibility.
							    
                    \item [Additional Comments: ]The framework of the session sets the stage for ample discussion time between panelists and attendees. The presenters will speak briefly to describe their roles in the project before opening the floor for questions from the audience. There will be a moderator to help facilitate the discussion. 

                \end{description}
              \subsection*{Audience}
                \begin{description}
                  \item [Audiences:]Curators/Scientists/Historians~Development and Membership Officers~Events Planning~Marketing \& Communications (Including Social Media)~Registrars, Collections Managers~Technology~
                  \item[Professional Level:]Emerging Professional~General Audience~Mid-Career~Senior Level~Student~
                \item[Scalability:] The process followed to develop Experience Oregon can be scaled up or down to fit any size organization. The size of the institution will determine the size of the advisory committee/stakeholder group. Having inspired, resolute staff and/or volunteers, a committed board, and dedicated consultants will be the recipe for success.

							
              \end{description}
            \subsection*{Participants}
              \subsubsection*{ Helen B Louise  }
              Submitter, Presenter\newline
              Museum Director \newline
              Oregon Historical Society , Portland, Oregon
              \newline
              Helen.Louise@ohs.org\newline
              
              503-306-5274\newline

              I am a panelist. My role was Project Manager for OHS on the Experience Oregon exhibition\newline


              
                \subsubsection*{ Andrew Hamilton }
                Moderator\newline
                Content Developer \newline
                The Design Minds , Fairfax, Virginia 
                \newline
                andrew@thedesignminds.com\newline
                
                703-246-9241\newline

                Andrew was a key member of the Experience Oregon team. He was the point of contact for The Design Minds and helped to develop the content. He worked closely with OHS and the stakeholders/advisory committee and tracked stakeholder comments.\newline
                \emph{ (confirmed) }
              

              
                \subsubsection*{ Darrell Millner  }
                Presenter\newline
                Professor Emeritus \& Adjunct Professor, Black Studies Department,\newline
                Portland State University , Portland, Oregon 
                \newline
                millnerd@pdx.edu\newline
                
                

                Darrell was on the stakeholder/advisory committee for the exhibit. He provided feedback on the content and ensured we were telling the most accurate story
                \emph{ (confirmed) }
              

              
                \subsubsection*{ Michael Lesperance  }
                Presenter\newline
                Principal/Content Developer \newline
                The Design Minds , Fairfax, Virginia 
                \newline
                Mike@thedesignminds.com\newline
                
                703-246-9241\newline

                Mike Lesperance is a Principal and Content Director at The Design Minds firm and the lead Project Manager for The Design Minds on this project.
                \emph{ (confirmed) }
              

              
                \subsubsection*{ Travis  Stewart  }
                Presenter\newline
                Artist/Interpretive Coordinator \newline
                Confederated Tribes of Grand Ronde , Grand Ronde, Oregon 
                \newline
                Travis.Stewart@grandronde.org\newline
                
                

                Travis was on the stakeholder/advisory committee. Travis is a member of the Confederated Tribes of Grand Ronde, one of the 9 federally recognized tribes in Oregon. He  contributed content to the exhibit and ensured OHS was telling the most accurate story for his tribe.
                \emph{ (confirmed) }
              

              
                \subsubsection*{ Amy  Platt }
                Presenter\newline
                Digital History Manager \newline
                Oregon Historical Society , Portland, Oregon 
                \newline
                Amy.Platt@ohs.org\newline
                
                503-306-5271\newline

                Amy was a key member of the content development team. Her deep knowledge of Oregon History ensured we were telling the most complete story from multiple perspectives
                \emph{ (confirmed) }
              
        
    \newpage
    \chapter*{ Half-day workshop (9:00 a.m. – 1:00 p.m.) }

      
        
        
    \newpage
    \chapter*{ Full-day workshop (9:00 a.m. – 4:00 p.m.) }

      
        
          \newpage
          \section{ Test }
            \begin{description}
              \item [ID:]
              WMA20201\_002

              \item [Assigned to:]Maren Dougherty~
                \item [Track:]
              \end{description}

              test2
\begin{itemize}
\item Bullet
\item Bullet 2
\item Bullet 3
Back to normal
\end{itemize}
Big-ass headline\newline

              \subsection*{Session Information}
                \begin{description}
                  \item [Format:] Full-day workshop (9:00 a.m. – 4:00 p.m.)
							    
							    \item [Uniqueness:]test whattahey.
							    \item [Objectives:]testnd
							    \item [Engagement:]test
							    \item [Relationship to Theme:]test
							    
                    \item [Additional Comments: ]test

                \end{description}
              \subsection*{Audience}
                \begin{description}
                  \item [Audiences:]Directors/Executive/C-Suite~Technology~
                  \item[Professional Level:]All professional levels~Emerging Professional~General Audience~
                \item[Scalability:]  test
 big-ish text
asdfasd

							
              \item[Other Comments:] sdfsdfsdsdfsdfsdf
              \end{description}
            \subsection*{Participants}
              \subsubsection*{ KOven Smith }
              Submitter\newline
              lkj\newline
              lkj, lkj
              \newline
              test@test.com\newline
              
              lkjlkj\newline

              


              

              

              

              

              
        \end{document}
