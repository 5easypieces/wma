\documentclass{report}

\title{ Western Museums Association Program Committee: Session Proposals }
\date{ Last updated: July 28, 2020}
\author{Western Museums Association}

\begin{document}
  \pagenumbering{gobble}
  \maketitle
  \newpage
  \tableofcontents
  \newpage
  \pagenumbering{arabic}
  
    \newpage
    \chapter*{ Regular session/panel (roundtable, single speaker, etc.) }

      
        
        
        
        
          \newpage
          \section{ Open Doors:  Making All Feel Welcome at a Niche Museum.  }
            \begin{description}
              \item [ID:]
              WMA2020\_SP34

              \item [Assigned to:]Doug Jenzen~
                \item [Track:]
              \end{description}

              Some museums naturally serve a very focused demographic in their region. Join representatives from 3 “niche” museums to learn about \$how they strive to stay true to their mission and core audience while making sure that all in their region feel a sense of welcome and belonging. This session will identify challenges and strategies to creating a welcoming environment and sense of belonging when your museum has a tight mission that serves a specific audience.

              \subsection*{Session Information}
                \begin{description}
                  \item [Format:] Regular session/panel (roundtable, single speaker, etc.)
							    
							    \item [Uniqueness:]This session tackles the difficult question of how niche museums welcome others into a largely mono-cultural community without compromising topical interests or diluting authenticity.
							    \item [Objectives:]- Empower other museums to embrace an “all are welcome” mentality.
- Identify challenges unique to “niche” museums that serve a very specific audience by default.
- Share strategies for creating welcome and belonging outside of a core-cultural audience.
							    \item [Engagement:]Engagement during the session will take the form of a slideshow presentation followed by a Q\&A. We will need a microphone (preferably cordless) for the audience if the session is recorded and/or amplified. We will also need a projector and screen for audio/visual presentations.
							    \item [Relationship to Theme:]-	Refining mission statements and values
-	Embracing diversity, equity, inclusivity, and accessibility
-	I also think this session fits with the theme of FORWARD in that it may provide insights into how we can build bridges in an era where people across the globe are becoming more and more polarized and insular.
							    
                    \item [Additional Comments: ]I have reached out to several people to be panelists. Currently, Gravity Goldberg (Director of Public Programs and Visitor Experience, The Contemporary Jewish Museum) has confirmed her interest and availability to be a panelist. Phillip Thompson (Executive Director, Idaho Black History Museum) has expressed interest. I have a few other irons in the fire as well. I am trying to find 3 panelists that will cover a few different perspectives on this topic as well as being representative of different types of museums at various scales. One area I am lacking panelists is around the issue of creating welcome beyond a group of core enthusiasts. (For our museum that might mean making "non-car-people" feel a sense of welcome and belonging at our motorsports museum.) I am open to suggestions and ideas for additional panelists that have had successes at their museum that they can share.

                \end{description}
              \subsection*{Audience}
                \begin{description}
                  \item [Audiences:]Curators/Scientists/Historians~Events Planning~Marketing \& Communications (Including Social Media)~
                  \item[Professional Level:]Emerging Professional~General Audience~Mid-Career~Senior Level~
                \item[Scalability:] My target audience are other museums that serve a very specific demographic in their region. However, I believe that the lessons learned at our session will still be applicable and beneficial to museums with a broader scope.

							
              \end{description}
            \subsection*{Participants}
              \subsubsection*{ Matthew Suplee }
              Submitter, Moderator, Presenter\newline
              Exhibit Developer\newline
              World of Speed, WILSONVILLE
              \newline
              matt@worldofspeed.org\newline
              msuplee@gmail.com\newline
              (225) 978-5554\newline

              I will mostly be moderating but I was planning on briefly presenting some of the challenges that our motorsports museum faces: bringing Diversity, Equity, Accessibility and Inclusion to a largely mono-cultural audience; creating appeal beyond just the motorsports community, and bridging diverse (and sometimes conflicting) socio-political viewpoints. I was going to use those challenges to set the stage for the panelists to present success stories from their museums on those fronts.\newline


              

              
                \subsubsection*{ Gravity Goldberg }
                Presenter\newline
                Director of Public Programs and Visitor Experience\newline
                The Contemporary Jewish Museum, San Francisco, CA
                \newline
                ggoldberg@thecjm.org\newline
                
                

                Successes in making non-Jewish persons feel welcome and included at CJM.
                \emph{ (confirmed) }
              

              
                \subsubsection*{ Phillip Thompson }
                Presenter\newline
                Executive Director\newline
                Idaho Black History Museum, Boise, ID
                \newline
                phillip.thompson@p-johnson.net\newline
                
                298-340-4448\newline

                I need to talk to him more but I am under the impression that he has worked to share the experience of being a black Idahoan in a region where the vast majority of people are not persons of color. I am curious how he has made inroads and what techniques he uses to reach people.
                \emph{ (not confirmed) }
              

              

              
    \newpage
    \chapter*{ Half-day workshop (9:00 a.m. – 1:00 p.m.) }

      
        
          \newpage
          \section{  Maximizing Social Media for Professional Development }
            \begin{description}
              \item [ID:]
              WMA2020\_WK2

              \item [Assigned to:]Jason Jones~
                \item [Track:]Technology~
              \end{description}

              Social platforms such as Twitter, LinkedIn, Facebook, and Reddit present evolving opportunities for museum professionals to share resources, ask questions, recognize trends, and amplify experiences for the benefit of the sector. This session will include strategies for finding peers to follow, hashtags to use, and groups that serve as gathering spaces for different areas of museum work. Participants will enhance their understanding of how to use social networking during and after the WMA Meeting.

              \subsection*{Session Information}
                \begin{description}
                  \item [Format:] Half-day workshop (9:00 a.m. – 1:00 p.m.)
							    
								  \item [Fee:]No
							     
							    \item [Uniqueness:]I do not think this type of workshop has taken place at previous WMA meetings. It will help participants maintain connections after the meeting ends.
							    \item [Objectives:]Participants will leave the workshop with an enhanced understanding of how to: 1) Use social media platforms to find and solicit information related to their area(a) of museum work; 2) Find, create, and/or join online groups related to their area(s) of work; 3) Publish information to larger circles of online audiences.  Participants will leave with an enhanced understanding of how to use social networking during the 2020 WMA Annual Meeting and after leaving Portland.
							    \item [Engagement:]Participants will be encouraged to log into their individual social media accounts at the start of the workshop in order to follow along as I show them how to find and solicit information on various platforms. They will be encouraged to ask questions, join groups, and publish posts during (or soon after) the session, depending on their comfort level.
							    \item [Relationship to Theme:]Our sector’s forward momentum depends on greater collaboration and communication. We are more efficient when we are constantly learning from one another, sharing resources, and amplifying our experiences for the benefit of the sector. Aspects of our work can be very similar but many of us work in small departments with modest budgets for training and research; social media is an inexpensive way of connecting people with questions to people with possible solutions.
							    
                    \item [Additional Comments: ]Happy to merge with another session and/or bring in additional speakers, depending on where this fits into the conference program.

                \end{description}
              \subsection*{Audience}
                \begin{description}
                  \item [Audiences:]Marketing \& Communications (Including Social Media)~Technology~
                  \item[Professional Level:]General Audience~
                \item[Scalability:] This topic is relevant for museum professionals at all types and sizes of organizations.

							
              \end{description}
            \subsection*{Participants}
              \subsubsection*{ Maren Dougherty }
              Submitter, Moderator, Presenter\newline
              EVP, Communications and Visitor Experience\newline
              Autry Museum of the American West, Los Angeles, CA
              \newline
              mdougherty@theautry.org\newline
              maren.dougherty@gmail.com\newline
              323-495-4259\newline

              If this fits into the program as a shorter workshop (e.g., lunchtime workshop), I could be the main presenter. If expanded, I could introduce the topics and ask social media savvy peers to offer case studies of how they have used specific social media platforms.\newline


              

              
                \subsubsection*{ Maren Dougherty }
                Presenter\newline
                EVP, Communications and Visitor Experience\newline
                Autry Museum of the American West, Los Angeles, CA
                \newline
                
                
                

                
                \emph{ (confirmed) }
              

              

              

              
        
          \newpage
          \section{ Visual Tools that Share Why Your Museum Matters  }
            \begin{description}
              \item [ID:]
              WMA2020\_WK4

              \item [Assigned to:]Jason Jones~
                \item [Track:]Business~
              \end{description}

              Why does your museum matter? How are you creating positive change in your community? This hands-on, half-day workshop will introduce museum professionals to a variety of tools that they can use to answer these questions for themselves, stakeholders, community members, and funders. Participants will learn how to use visual Theories of Change, Theories of Action, and Logic Models to capture “why” and “how” they are making a difference.

              \subsection*{Session Information}
                \begin{description}
                  \item [Format:] Half-day workshop (9:00 a.m. – 1:00 p.m.)
							    
								  \item [Fee:]I would like to keep the fee minimal while covering the cost of materials such as sticky notes, chart packs, markers, and print outs for people to use at the workshop and take back to their institutions. I suggest 10/person. 
							     
							    \item [Uniqueness:]Visual impact models help museums articulate their value in ways that engage and inspire stakeholders, staff, partners, and funders.
							    \item [Objectives:]Participants will:
1. Understand how to use a Theory of Change, Theory of Action, and Logic Model to articulate organizational impacts.
2. Create draft versions of each tool to take back to their institutions and adapt for their programs.
3. Return to their museums with at least one sample activity/facilitation technique that they can lead to develop each tool further with their staff, communities, and stakeholders.
							    \item [Engagement:]The workshop will focus on interactive visual thinking strategies including:
•	Illustrate the museum’s vision for how positive change happens and how their institution supports this process.
•	Group “Storyboarding” to identify logical connections between audiences, museum/community needs, desired impacts, and exhibits/programs.
•	Discussions by museum type to articulate their “9 Whys”—why their institutions exist and how they benefit their communities.

Resources needed: projector/screen, tables, chairs, chart paper, markers, sticky notes, evocative images/icons, creativity, willingness to participate
							    \item [Relationship to Theme:]How do we move forward without a vision for change? How do excite stakeholders and funders to support us without clearly articulating our vision? This workshop helps people develop the tools that they need to clarify and share their impact, allowing them to move forward.
							    
                    \item [Additional Comments: ]I am an independent museum professional (aka, a consultant). I am not expecting to be paid for my work to prepare or present the workshop, nor am I doing this to sell my services. I was asked by a colleague to propose this workshop so that more museums, especially rural museums, smaller museums, and museum staff asked to take on fundraising/engagement responsibilities, could learn more tools for sharing why their institutions matter. That said, I will be using my work as an example in the workshop and hope that participants will see me as a resource afterwards. If the review committee has any questions or suggestions about how the workshop can provide the most benefit to participants or potential co-presenters, I’d love your input.

                \end{description}
              \subsection*{Audience}
                \begin{description}
                  \item [Audiences:]Curators/Scientists/Historians~Development and Membership Officers~Marketing \& Communications (Including Social Media)~
                  \item[Professional Level:]Mid-Career~Senior Level~
                \item[Scalability:] Visual impact models can be used by any type and size of museum for a variety of purposes from strategic planning, interpretive planning, and fundraising.

							
              \end{description}
            \subsection*{Participants}
              \subsubsection*{ Kyrie Kellett }
              Submitter, Moderator, Presenter\newline
              Principal\newline
              Mason Bee Interpretive Planning, Portland, OR
              \newline
              connect@masonbeellc.com\newline
              kyrie.kellett@gmail.com\newline
              503-419-7735\newline

              I will facilitate the workshop. I have led several organizations through the process of creating visual impact tools, and look forward to sharing what I've learned with other museum professionals.\newline


              

              
                \subsubsection*{ Kyrie Kellett }
                Presenter\newline
                Principal\newline
                Mason Bee Interpretive Planning, Portland, OR
                \newline
                connect@masonbeellc.com\newline
                kyrie.kelllett@gmail.com\newline
                503-419-7735\newline

                I was asked by colleagues to share what I know about how to create visual impact models for museums.
                \emph{ (confirmed) }
              

              

              

              
    \newpage
    \chapter*{ Full-day workshop (9:00 a.m. – 4:00 p.m.) }

      
        
        
          \newpage
          \section{ test }
            \begin{description}
              \item [ID:]
              WMA2021\_001

              \item [Assigned to:]Kathleen Daly~
                \item [Track:]Indigenous~
              \end{description}

              \begin{enumerate}
\item Session description! Here it is!
\end{enumerate}
A big headline\newline
\begin{enumerate}
\item An ordered list
\item The second item in the ordered list
\item The third item
\end{enumerate}
More paragraph content and such.

              \subsection*{Session Information}
                \begin{description}
                  \item [Format:] Full-day workshop (9:00 a.m. – 4:00 p.m.)
							    
								  \item [Fee:]5.00
							     
							    \item [Uniqueness:]test
							    \item [Objectives:]testing
							    \item [Engagement:]test
							    \item [Relationship to Theme:]asdfasdfsad
							    
                \end{description}
              \subsection*{Audience}
                \begin{description}
                  \item [Audiences:]Facilities Management Personnel~Marketing \& Communications (Including Social Media)~Technology~
                  \item[Professional Level:]All professional levels~General Audience~
                \item[Scalability:] asafdsfasdasdas
asdasdasd
asdasdasdas

							
              \item[Other Comments:] asdfasdfsdfsdfsdsdfsdfsdf
              \end{description}
            \subsection*{Participants}
              \subsubsection*{ Koven Smith }
              Submitter, Presenter\newline
              title\newline
              org, Austin
              \newline
              fuf\newline
              koven.smith@gmail.com\newline
              9178038432\newline

              Justy\newline


              
                \subsubsection*{ test test }
                Moderator\newline
                yep\newline
                inst, city
                \newline
                test@test\newline
                
                tel\newline

                mod\_just\newline
                \emph{ (confirmed) }
              

              
                \subsubsection*{ Smith Koven }
                Presenter\newline
                asdf\newline
                asdf, asdf
                \newline
                koven@kovenjsmith.com\newline
                
                asdf\newline

                asadfsadf
                \emph{ (not confirmed) }
              

              

              

              \end{document}
